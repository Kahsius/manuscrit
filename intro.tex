\chapter{Abstract}

This thesis presents several unsupervised algorithms used in a collaborative context. Its main axis is the improvement of communications between several distributed datasets and the developments of these communications, either in terms of quality or in terms of use case. This axis is then divided into two sub problems, each one defined by the goal of the algorithm.

The first one is Collaborative Clustering, which aims at refining local results through a consensus between several distributed datasets called views. Given a local view, this consensus is achieved through the prioritization of the information got from all the external views. The method presented in this thesis is generic and can be applied without consideration of the clustering algorithms used on each view.

The second use case is the reconstruction of missing data. Given several views, and considering a set of individuals described in more than one view, the goal of this use case is to transfer information from a view to another in order to infer an approximation of the missing description of an individual. While Collaborative Clustering has already been extensively studied in the literature, Collaborative Reconstruction is introduced in this thesis. This new use case is applied on several datasets containing either images or numerical vectors, the majority of which being commonly found in Collaborative Clustering literature. However, our method differs from what has previously been presented in the literature because the inter-views communication is performed both by a scalar based prioritization of the information and by a neural network based inference system. Moreover, while Collaborative Clustering ensures a minimum security by transfering informations different from the original data, our Collaborative Reconstruction method uses autoencoders, a specific kind of neural networks, to encode informations before sending them to ensure a minimum security regarding data transfer.

\chapter{Introduction}

\minitoc{}
\newpage

\section{Thesis Scope}

Each day, data grows in terms of quantity and complexity. This has the advantage of making possible a wide range of applications on different data of all natures, but it also has the disadvantage of making each application more specific, and their possible solutions more difficult to design. Moreover, because of this huge volumetry, one is likely to face a problem for which data are distributed among several datasets (also called views). It is in this context that Multi-View Learning has been created. This latter can be divided into several paradigms focused on the collaboration between different actors to fulfill different goals. Among them, two paradigms are specificaly focused on clustering: Collaborative and Cooperative Clustering.

Clustering is the process of organizing individuals in groups such as the similarity between the individuals of a same group is maximum while the similarity between individuals from two different groups is minimum. However, it is an ill-posed problem because there is no universal definition of what a similarity measure is and the determination of the right number of clusters might be difficult. Multi-view Clustering aims at mitigating the problem inherent to each local clustering by establishing inter-views communications.

Cooperative Clustering aims at finding a global consensus knowing groups identified locally obtained by each view, while Collaborative Clustering aims at updating local results to take into account locally groups that have been identified in the other views. As part of this thesis, a consensus is defined as the modification of each local solution to maximize the inter-views agreement while not hindering local results. Collaborative Clustering is sometimes considered as a specific phase of Cooperative Clustering, just before merging the results of each view to get the global consensus. These two paradigms being vast research fields in themselves, in this thesis we focus on horizontal Collaborative Clustering, the subfield considering several views describing the same set of individuals using different feature sets. In this context, the goal of Collaborative Clustering is to consider groups identified from different point of views (hence the designation of view) to update each local clustering. In that sense, one could consider Collaborative Clustering to look for a consensus. However, it is important to note that, while Cooperative Clustering aims at finding a unique global consensus, Collaborative Clustering rather aims at improving each local clustering to ensure that local results take into consideration the information extracted by all the other clusterings.

Considering the fundamental idea of Collaborative Clustering, namely the collaboration of several independent views to get local results, we have explored the application of inter-views communications to Collaborative Reconstruction. Having several datasets describing the same set of individuals but using different sets of features might sometimes implies that some individuals may be missing in some datasets while being present in some other. It might be because data have been gathered using various methods during different sessions maybe separated in time, or because individuals did not provide all the required data during a survey for example. 

In this context, the information regarding a specific individual and spread among several external views could be used to infer a local approximation of this individual. This thesis present a system able to perform such inferences. An interest in this use case is to consider inter-views communications from an other point of view than the clustering one. To do so, the initial hypothesis of the shared set of individuals is used to infer links between the representations of an individual in two different feature spaces. In other words, we want to make a system learn the correspondences between two different features spaces using the individuals that these features spaces have in common.

To sum up, the improvement of inter-views communications in a collaborative context is developped either based on existing methods in the case of Collaborative Clustering, or on the definition of a new paradigm in the case of Collaborative Reconstruction.

\section{Thesis Overview}

This thesis is structured into four chapters (Introduction, Conclusion and Appendices excluded) and is organized as follows:\\

\textbf{Chapter 2, State of the Art:} This chapter introduces a state of the art divided in four main parts. First, a definition of clustering and its current research challenges are provided. Then the most commonly found methods are presented, to finally introduce Collaborative Clustering as topic in itself. The aim of this chapter is to bring the reader from a high level definition of what the clustering is, to a more specific collaborative context.\\

\textbf{Chapter 3, Incremental Self Organizing Maps based Collaborative Clustering:} This chapter present our early work on multi-view communications. The objective was to define a method train Self Organizing Maps in both a collaborative and incremental context. It is related to the problem of communications in a collaborative context because it focuses on how these communications could be performed all along the training process, and not just during a fixed segment in time. Thus the chapter first introduces the problem of online training in a collaborative context. Based on the limitations identified, we present an online version of Self Organizing Maps. This later method is then adapted to be used in a Collaborative Context. Experiments are presented to determine the efficiency of our new Collaborative Clustering method. A discussion regarding the limitations of this method as well as a conclusion ends the chapter.\\

\textbf{Chapter 4, Optimized weights for the horizontal collaborative SOM algorithm:} Based on the challenges and on the state of the art presented previously, this chapter presents a new weighting method used to define the relative confidence a view has in the information coming from all its external peers. We first define our problem as an optimization problem, leading to the creation of a constrained system of equation defining the values of the best collaboration coefficients. This system is then solved mathematically using the Karush-Kuhn-Tucker method, followed by an interpretation of the results. The method is finally numerically against existing approaches in the literature.

\textbf{Chapter 5, Collaborative Reconstruction System:} This chapter presents our method to achieve Collaborative Reconstruction. To the best of our knowledge, there is no existing work on Collaborative Reconstruction in the literature, thus we first give some definitions. Then, we introduce the reader to the main components used by the system, a quick overview of Neural Networks is then given with a focus on two of the many kinds of existing Neural Networks: Multi-Layer Perceptrons (MLP) and Autoencoders. After this introduction, a global definition of the architecture of what we call the Collaborative Reconstruction System is given. The main components are the Autoencoders that encode the information, the MLP which transfer the information from a view to an other, and the Masked Weighting Method which combines all the external informations. This later component being a contribution in itself, its definition and its training are detailed using a mathematical analysis of the learning process using either a Gradient Descent or an iterative process. Because of the complexity of the system, several sets of experiments have been conducted to test both the efficiency of the global system and the impact of this method on the results of the system. A graphical representation of what can be achieved in terms of reconstruction is presented using a dataset of handwritten digits. Finally, a discussion of the advantages and on the limits of the system is presented at the end of the chapter.\\


\section{Main Contributions}

\textbf{\\International Journal}

\begin{itemize}
    \item (currently submited) Denis Maurel, Sylvain Lefebvre and J{\'{e}}r{\'{e}}mie Sublime,
        \textit{Deep Cooperative Reconstruction with Security Constraints},
        Knowledge And Information System (KAIS),
        2019.
\end{itemize}

\textbf{\\International Conferences}

\begin{itemize}
    \item Denis Maurel, J{\'e}r{\'e}mie Sublime and Sylvain Lefebvre,
        \textit{Incremental Self-Organizing Maps for Collaborative Clustering},
        International Conference on Neural Information Processing,
        2017.

    \item J{\'e}r{\'e}mie Sublime, Denis Maurel, Nistor Grozavu, Basarab Matei and Younès Bennani,
        \textit{Optimizing exchange confidence during collaborative clustering},
        International Joint Conference on Neural Networks (IJCNN),
        2018.
\end{itemize}

\textbf{\\National Conference}

\begin{itemize}
    \item Denis Maurel, J{\'e}r{\'e}mie Sublime and Sylvain Lefebvre,
        \textit{Cartes Auto-Organisatrices Incrémentales appliquées au Clustering Collaboratif},
        Conf{e}rence internationale sur l'extraction et la gestion des connaissances,
        2017.
\end{itemize}
