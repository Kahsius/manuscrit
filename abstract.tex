%        File: abstract.tex
%     Created: lun. sept. 10 10:00  2018 C
% Last Change: lun. sept. 10 10:00  2018 C
%

\documentclass[a4paper]{article}
\usepackage[utf8]{inputenc}
\usepackage[french]{babel}
\usepackage[T1]{fontenc}

\begin{document}
\title{Contributions aux communications inter-vues pour l'apprentissage collaboratif}
\author{Denis Maurel}
\maketitle

\textbf{Mots clés :} \quad Apprentissage non-supervisé, clustering collaboratif, reconstruction collaborative\\%chktex 26


\textbf{Résumé :} \quad Cette thèse présente plusieurs méthodes d'optimisation et d'amélioration des communications inter-vues dans un contexte d'apprentissage collaboratif. Deux axes sont développés:\\%chktex 26

Le premier concerne l'amélioration des communications pour le clustering collaboratif, un paradigme dans lequel plusieurs jeux de données, appelés vues, sont utilisés pour effectuer un premier clustering local avant de s'échanger des informations afin de parvenir à un concensus sur leurs résultats. Notre premier contribution consiste en une méthode d'apprentissage permettant à une vue de pondérer l'information fournit par les vues externes. Cette méthode se base sur la résolution d'un problème constitué du critère du clustering collaboratif auquel à été ajouté deux contraintes sur les coefficients de pondérations.

Une seconde contribution consiste en la définition d'une méthode d'apprentissage incrémentale de cartes auto-adaptatrices de Kohonen, suivie de son adaptation au clustering collaboratif. Cette méthode permet entre autre la mise à jour des résultats obtenus via le clustering collaboratif en cas d'évolution dans la distribution des données pouvant survenir au cours du temps.\\

Le second axe consiste en la définition d'un nouveau paradigme collaboratif, appelé reconstruction collaborative. Dans ce paradigme, plusieurs vues collaborent pour reconstruire des données localement manquantes. Cette méthode se base sur des réseaux de neurones permettant de faire le lien entre les données externes et les données locales. La combinaison des informations externes est assurée par une méthode de pondération permettant de privilégier les caractéristiques les mieux reconstruites par chaque vue externe.
\end{document}


