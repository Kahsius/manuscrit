%        File: abstract.tex
%     Created: lun. sept. 10 10:00  2018 C
% Last Change: lun. sept. 10 10:00  2018 C
%

\documentclass[a4paper]{article}
\usepackage[utf8]{inputenc}
\usepackage[french]{babel}
\usepackage[T1]{fontenc}

\begin{document}
\title{Contributions to inter-views communications applied to collaborative learning}
\author{Denis Maurel}
\maketitle

\textbf{Mots clés:} \quad Unsupervised learning, Collaborative Clustering, Collaborative Reconstruction\\


\textbf{Résumé:} \quad This thesis presents several methods to optimize and improve inter-views communications in a collaborative learning context. Two axis are developed:\\

The first one is about the improvement of communications for Collaborative Clustering, a paradigm for which several datasets, called views, are used to perform a first local clustering before exchanging informations in order to get a consensus on their results. Our first contribution consists in a learning method making it possible for a view to weight the information supplied by the external views. This methods is based on the resolution of a problem made of the Collaborative Clustering criterion with two constraints of the weighting coefficients.

A second contribution consists in the definition of an incremental learning method of Self-Organizing Maps, followed by its adaptation to Collaborative Clustering. This method makes it possible to adapt the results obtained using Collaborative Clustering in case of a potential evolution in data distribution through time.\\

The second axis consists in the definition of a new paradigm, called Collaborative Reconstruction. In this paradigm, several views collaborate to reconstruct local missing data. This method is based on neural networks linking external data and local data. The combination of the external informations is guaranteed by a weighting method favoring the best reconstructed features for each external view.
\end{document}


